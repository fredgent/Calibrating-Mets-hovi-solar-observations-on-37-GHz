
  %FAG: reform
  %We then solve the real roots of $P^{\prime\prime}(u)$, to obtain 
  %\eqnl{scurve-roots}{
  %u_1 < ... < u_{n-2} \quad \text{, for which} \quad P^{\prime\prime}(u_j) = 0\text{.}
  %}
  %We then solve the real roots of $P^{\prime\prime}(u)$, to obtain 
  %$u_1 < ... < u_{n-2}$, for which $P^{\prime\prime}(u_j) = 0$.
  %There is an odd number of such roots, so we choose the middle root:
  %$u_0 := u_{{(n-1)}/{2}}$.
  
  %FAG: sample->specimen
  %The sample $f_S(i_0)$ with index $i_0 := \floor{w(u_0)+0.5}$ is essentially in the middle of the disk boundary and provides pivot for signal level references. We set:
  The specimen $f_S(i_0)$ with index $i_0 = \floor{w(u_0)+0.5}$ is essentially
  in the middle of the disk boundary and provides a pivot for signal level
  references.
  We set:
  \eqnl{scurve-pivots}{
  i_1 = \floor{\frac{i_0 + 1 + 1}{3} + 0.5} \quad \text{and} \quad
  i_2 = \floor{\frac{i_0 + |S|}{2} + 0.5} \text{.}
  }
  The intensity $w(i_1)$then represents the background sky and $w(i_2)$ QSL.
  %FAG: reworded
  %We then redo the polynomial fitting only for those samples which have intensity in the dynamic range of disk boundary, $10 .. 90 \%$ of QSL.
  We then apply the polynomial fitting only for those specimens within the
  disk boundary intensity dynamic range $10 .. 90 \%$ of QSL.
  \eqnl{scurve-approx1}{
  P^{(1)}(u) &\approx& w(u) \quad \text{for}\\ \quad \frac{9 w(i_1) + w(i_2)}{10} &\le u \le& \frac{w(i_1) + 9 w(i_2)}{10} \text{.} \nonumber
  }
  The result of this second fitting is shown in
  Fig.~\ref{S-curve_example}{\bf(a)}. 
  %FAG: out of math mode
  We find the roots $u_1^{(1)}< ... < u_{n-2}^{(1)}$, for which
  $P^{(1)\prime\prime}(u_j^{(1)}) = 0$, and $u_0^{(1)} = u_{{(n-1)}/{2}}^{(1)}$ 
  and set the pivot indices 
  %We find the roots:
  %\eqnl{scurve-roots1}{
  %%MJK One line
  %%u_1^{(1)} < ... < u_{n-2}^{(1)} 
  %&u_1^{(1)}& < ... < u_{n-2}^{(1)} \text{, for which} \quad \\
  %%MJK to remove space
  %%\quad 
  %%\text{, for which} \quad
  %&P^{(1)\prime\prime}(u_j^{(1)})& = 0 \text{, and} \quad 
  %u_0^{(1)} := u_{\frac{n-1}{2}}^{(1)} \nonumber 
  %}
  %%MJK indices?
  %and set pivot index
  %MJK
  as
  %:
  \eqnl{scurve-pivots1}{
  %MJK One line
  &i_0^{(1)}& = \floor{w(u_0^{(1)})+0.5}
  %MJK comma
  \text{,}
  \quad i_1^{(1)} = \floor{\frac{i_0^{(1)} + 1 + 1}{3} + 0.5} 
  %\quad 
  %MJK comma
  \text{,} 
  %and} 
  \nonumber \\
  %\quad 
  &i_2^{(1)}& = \floor{\frac{i_0^{(1)} + |S|}{2} + 0.5} \text{.}
  }
  %FAG: introduce $v$
  %We can now calibrate the intensity using $w(i_1^{(1)})$ for the background and $w(i_2^{(1)})$for QSL:
  We can now calibrate the normalised intensity, $v(s)$,
  using $w(i_1^{(1)})$ for the background and $w(i_2^{(1)})$for QSL:
  \eqnl{scurve-calibration}{
  s \in S \implies v(s) = \frac{u(s) - w(i_1^{(1)})}{w(i_2^{(1)}) - w(i_1^{(1)})} \text{.}
  }


  \subsubsection{2 Convex hull}\label{sect:convex-hull}

  The convex hull method was mainly used for the contour maps for increased redundancy. Should the other markings on the 
  maps be misinterpreted, the circular limb profile gives a hint of where the solar disk should be. The convex hull 
  suits for digital intensity specimen as well.

  We separate the set of points $S = A \cup B$ with $A \cap B = \emptyset$, where $A$ 


  \fag{added subsection for convex hull. Is the previous subsection complete?
  It is not clear to me how disk fitting (1) is explained in previous
  subsection.
  Does the text below apply to this method (2)?}
  For method (1) the sample observations in each map, as shown in
  Fig.~\ref{oldmap}, include white specimens, which are excluded from the disk
  fitting, and red specimens, which are included based on a limb model.
  Green dots show radial offset correction to the red specimens, which constitute
  the target set for circle fitting.
  \fag{What is meant by radial offset? To which limb model do we refer?}
  %\fag{Comment: Explain what is added up as part of calibration. Include another plot with the final version in Solar coordinates.}

  %FAG reordered text, but still edit TODO
  %Determine the solar disk position and size from each map as well as
  %normalise the measured background intensity for zero and the Quiet Sun
  %Level (QSL) for unity.
  %This can be done using a binary combination of two disk methods (1 and 
  %2) and three normalisation methods (a, b, and c).
  %The methods are: (1) intentionally weighted center of mass of the radio
  %sample coordinates, (2) fitting a circle for those sample coordinates
  %which we, by geometry, expect to belong to the solar limb, (a) sort the
  %intensity samples and detect the point of zero curvature from the index vs.     intensity dependency, the S-curve, (b) iteratively calculate the
  %statistical mean and standard deviation of the intensity of 
  %those points which are located within the solar disk and not too close
  %to the limb, (c) detect two significant peaks from the intensity histogram,
  %the lower intensity peak being the background and the higher the QSL.

  Method (2) can be augmented using two additional sets of information.
  When we observe exactly at the limb, we would expect $0.5$ QSL intensity
  due the the beam mixing disk and background signal.
  If one particular sample has intensity $0.4$ QSL, we would expect the
  sample to be slightly more away from the center.
  Collecting a large set of observations on different conditions, we can
  estimate how quickly the intensity drops as the beam passes the limb 
  from disk to background.
  Utilizing this limb profile, we can fine tune the position of each limb
  sample in the target set and proceed iteratively to obtain more
  accurate circle fitting.
  
  %FAG: Following tract from intro now merged into previous paragraphs
  %%During the 40 years, the equipment 
  %%has experienced several upgrades and there has also been changes in the data format.
  %%The data requires careful 
  %%normalisation procedures, after which the butterfly diagram can be constructed from the solar maps.
  %%The solar maps from 
  %%1978 to 1987 were originally recorded on magnetic tapes, out of which the data was rendered on contour maps using a 
  %%mechanical plotter. Since then, the magnetic tapes have been lost, and the maps were only available as 
  %%%MJK it would not hurt to stress that they were printer on paper only -> plots -> printouts.
  %%these plots.
  %%%MJK or rather pattern recognition?
  %%Using various image recognition techniques featured in \cite{masterthesis}, these contour maps were converted back into 
  %%numerical form. For these prepared contour maps, 
  %%%MJK Define better what you mean by 'solar disk being available'. It might not be clear to a common reader.
  %the solar disk was already available, while for the later maps from 
  %1989 to present the disk 
  %has to be detected using automatic image recognition. 
  %%MJK To make it more clear:
  %%MJK The purpose of this paper is to present two methods () for detecting the disk and ....
  %We present two methods (1: S-curve and 
  %center of mass, 2: convex hull) for detecting the disk and three methods (a: S-curve and constant fraction, b: 
  %statistics and outlier neglection, c: interpolating between histogram peaks) for normaliseing the signal levels so that 
  %zero is the black sky at the background and unity is the Quiet Sun Level (QSL).
  %%MJK Then there should be a sentence stating that "As our final data product we present here a butterfly diagram of radio brightenings extending over the full time span of the recorded Mets\"ahovi solar data set".

%M%JK This would be better in the methods section, to outline the general workflow from the raw data to the final product.
  %We recognize six suitable combinations of options for the disk model $F_j$ in 
  %%MJK A&A style
  %%Equation \ref{disk_model}. 
  %Eq.~\ref{disk_model}. 
  %We can 
  %determine the disk size and position using either (1) histogram S-curve (2) convex hull based approach. Furthermore, the 
  %signal levels within the disk can be calibrated by (a) using constant histogram fractions, (b) iteratively calculating 
  %standard deviation, outliers excluded, or (c) interpolating between two slots in a histogram peak. In this study, we have compared the method 
  %combinations $j = \mathrm{1a}, \mathrm{2b}, \mathrm{2c}$.
  %%MJK I guess there is a reason(s), why you have converged to these combinations. Hence, it would be better to say: We investigated all the possible combinations of these methods, and due to following reasons converged to the combinations 1a, 2b and 2c being the optimal ones: blaa blaa. Here we present only comparisons in between these optimal method combinations.
  
  To compare these calibration methods, we arbitrarily choose an interval which contains $n$ solar maps within a 
  reasonable time span $[\s{t}{min},\s{t}{max}]$. We process all these maps using each method $j$, and produce a list of 
  trackable radio features $F_j$. Due to differences in the operation of each calibration method, we do not expect the 
  contents of $F_j$ to be identical for each $j$. Nevertheless, we can pick $m$ features which visually appear to be in 
  the same location in all $F_j$.
  
  Arbitrary selected active regions $\left\{ \right\}_{k=1}^{5}$ are expected to be detected on maps produced in each of 
  %MJK please select one notation, either with or without brackets. I like the bracketed version more.
  the fitting models (1a), (2b), and (2c). When seen from Earth, the active region makes a steady progress from 
  %MJK No 'the', Fred?
  the east 
  to west. When these observations on relative $(\mathrm{RA}, \mathrm{dec}$ coordinates are plotted on heliographic 
  surface, we expect them to fit on a great
  %MJK commas required
  circle. The quality of the fitting algorithm $F_j$ can thus be measured as the 
  inverse of standard deviation from the best agreeing isocircle obtained using the 
  %MJK hyphen
  least squares method.
%MJK Is there supposed to be a section about the convex hull method?

